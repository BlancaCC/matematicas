
\ifdefined\COMPLETE

%%%%%%%%%%%%%%%%%%%%% comienzo del documento%%%%%%%
\begin{itemize}
	\item \textbf{Recocorrido o rango}
	\par Supuesto de orden creciente R = $ x_k - x_1$
	\item \textbf{Recorrido intercuartílico }
	\par Misma idea anterior $R_i = Q_3 - Q_1$ Longitud del interval donde están incluidos el 50 \% de los datos. 
	\item \textbf{Desviación absoluta respecto de a }
	$$ D_a = \frac{1}{n} \sum_{i=1}^{k} | x_i - a| n_i$$
	Se minimiza en la mediana.
	
	\item \subsection*{Varianza} 
	$$ D_a = \frac{1}{n} \sum_{i=1}^{k} \big( x_i - a \big)^2 n_i$$
	\begin{itemize}
		\item {$\cdot$}Nunca puede ser negativa
		\item La varianza de a= $\bar{x}$ es óptima. 
		\item Teoremilla de Köning 
	\end{itemize}
	
	
\end{itemize}


%%%%%%%%%%%% fin de vuestro documento %%%%%%%%%%%%%%%%%%%%%%%%%%%%%%%%
\else 
\def\COMPLETE{}

\documentclass[a4paper , 11pt, spanish ]{article}
% Codificación e idioma, para las tildes crucial
\usepackage{paquetes}
\title {Medidas de dispersión}
\author{Blanca}


\begin{document}
\maketitle 

\ifdefined\COMPLETE

%%%%%%%%%%%%%%%%%%%%% comienzo del documento%%%%%%%
\begin{itemize}
	\item \textbf{Recocorrido o rango}
	\par Supuesto de orden creciente R = $ x_k - x_1$
	\item \textbf{Recorrido intercuartílico }
	\par Misma idea anterior $R_i = Q_3 - Q_1$ Longitud del interval donde están incluidos el 50 \% de los datos. 
	\item \textbf{Desviación absoluta respecto de a }
	$$ D_a = \frac{1}{n} \sum_{i=1}^{k} | x_i - a| n_i$$
	Se minimiza en la mediana.
	
	\item \subsection*{Varianza} 
	$$ D_a = \frac{1}{n} \sum_{i=1}^{k} \big( x_i - a \big)^2 n_i$$
	\begin{itemize}
		\item {$\cdot$}Nunca puede ser negativa
		\item La varianza de a= $\bar{x}$ es óptima. 
		\item Teoremilla de Köning 
	\end{itemize}
	
	
\end{itemize}


%%%%%%%%%%%% fin de vuestro documento %%%%%%%%%%%%%%%%%%%%%%%%%%%%%%%%
\else 
\def\COMPLETE{}

\documentclass[a4paper , 11pt, spanish ]{article}
% Codificación e idioma, para las tildes crucial
\usepackage{paquetes}
\title {Medidas de dispersión}
\author{Blanca}


\begin{document}
\maketitle 

\ifdefined\COMPLETE

%%%%%%%%%%%%%%%%%%%%% comienzo del documento%%%%%%%
\begin{itemize}
	\item \textbf{Recocorrido o rango}
	\par Supuesto de orden creciente R = $ x_k - x_1$
	\item \textbf{Recorrido intercuartílico }
	\par Misma idea anterior $R_i = Q_3 - Q_1$ Longitud del interval donde están incluidos el 50 \% de los datos. 
	\item \textbf{Desviación absoluta respecto de a }
	$$ D_a = \frac{1}{n} \sum_{i=1}^{k} | x_i - a| n_i$$
	Se minimiza en la mediana.
	
	\item \subsection*{Varianza} 
	$$ D_a = \frac{1}{n} \sum_{i=1}^{k} \big( x_i - a \big)^2 n_i$$
	\begin{itemize}
		\item {$\cdot$}Nunca puede ser negativa
		\item La varianza de a= $\bar{x}$ es óptima. 
		\item Teoremilla de Köning 
	\end{itemize}
	
	
\end{itemize}


%%%%%%%%%%%% fin de vuestro documento %%%%%%%%%%%%%%%%%%%%%%%%%%%%%%%%
\else 
\def\COMPLETE{}

\documentclass[a4paper , 11pt, spanish ]{article}
% Codificación e idioma, para las tildes crucial
\usepackage{paquetes}
\title {Medidas de dispersión}
\author{Blanca}


\begin{document}
\maketitle 

\ifdefined\COMPLETE

%%%%%%%%%%%%%%%%%%%%% comienzo del documento%%%%%%%
\begin{itemize}
	\item \textbf{Recocorrido o rango}
	\par Supuesto de orden creciente R = $ x_k - x_1$
	\item \textbf{Recorrido intercuartílico }
	\par Misma idea anterior $R_i = Q_3 - Q_1$ Longitud del interval donde están incluidos el 50 \% de los datos. 
	\item \textbf{Desviación absoluta respecto de a }
	$$ D_a = \frac{1}{n} \sum_{i=1}^{k} | x_i - a| n_i$$
	Se minimiza en la mediana.
	
	\item \subsection*{Varianza} 
	$$ D_a = \frac{1}{n} \sum_{i=1}^{k} \big( x_i - a \big)^2 n_i$$
	\begin{itemize}
		\item {$\cdot$}Nunca puede ser negativa
		\item La varianza de a= $\bar{x}$ es óptima. 
		\item Teoremilla de Köning 
	\end{itemize}
	
	
\end{itemize}


%%%%%%%%%%%% fin de vuestro documento %%%%%%%%%%%%%%%%%%%%%%%%%%%%%%%%
\else 
\def\COMPLETE{}

\documentclass[a4paper , 11pt, spanish ]{article}
% Codificación e idioma, para las tildes crucial
\usepackage{paquetes}
\title {Medidas de dispersión}
\author{Blanca}


\begin{document}
\maketitle 
\input{./Medidas_de_dispersion} %nombre de este documento 
\end{document}


\fi
 %nombre de este documento 
\end{document}


\fi
 %nombre de este documento 
\end{document}


\fi
 %nombre de este documento 
\end{document}


\fi
